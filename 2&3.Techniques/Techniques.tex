\newpage
\section{Techniques}

\subsection{Compile-Time Assertions}

\textbf{C++17 provides \texttt{static\_assert}}.

The simplest solution to compile-time assertions works in C as well as in
C++, relies on the fact that a zero-length array is illegal.

\begin{verbatim}
#define STATIC_CHECK(expr) { char unnamed[(expr) ? 1 : 0]; }
template <class To, class From>
To safe_reinterpret_cast(From from){
  STATIC_CHECK(sizeof(From) <= sizeof(To));
  return reinterpret_cast<To>(from);
}
void* somePointer = ...;
char c = safe_reinterpret_cast<char>(somePointer);
\end{verbatim}

The problem with this approach is that the error message you receive
is not terribly informative. Error messages have no rules that they
must obey; it's all up to the compiler.

A better solution is to rely on a template with an informative name;
with luck, the compiler will mention the name of that template in the
error message.

\begin{verbatim}
template<bool> struct CompileTimeError;
template<> struct CompileTimeError<true> {};
#define STATIC_CHECK(expr) \
(CompileTimeError<(expr) != 0>())
\end{verbatim}

If you try to instantiate
\texttt{CompileTimeError<false>}, the compiler utters a message such
as "Undefined specialization \texttt{CompileTimeError<false>}." This
message is a slightly better hint that the error is intentional and
not a compiler or a program bug.

Actually, the name \texttt{CompileTimeError} is no longer suggestive
in the new context. \textbf{The ellipsis means the constructor accepts anything.}

\begin{verbatim}
template<bool> struct CompileTimeChecker{
  CompileTimeChecker(...);
};
template<> struct CompileTimeChecker<false> { };
#define STATIC_CHECK(expr, msg) {\
  class ERROR_##msg {}; \
  (void)sizeof(CompileTimeChecker<(expr) != 0>((ERROR_##msg())));\
}

template <class To, class From>
To safe_reinterpret_cast(From from){
  STATIC_CHECK(sizeof(From) <= sizeof(To),Destination_Type_Too_Narrow);
  return reinterpret_cast<To>(from);
}
void* somePointer = ...;
char c = safe_reinterpret_cast<char>(somePointer);
\end{verbatim}

After macro preprocessing, the code of \texttt{safe\_reinterpret\_cast}
expands to the following:

\begin{verbatim}
template <class To, class From>
To safe_reinterpret_cast(From from){
  class ERROR_Destination_Type_Too_Narrow {};
  (void)sizeof(
    CompileTimeChecker<(sizeof(From) <= sizeof(To))>(
      ERROR_Destination_Type_Too_Narrow()));
  return reinterpret_cast<To>(from);
}
\end{verbatim}

The \texttt{CompileTimeChecker<true>} specialization has a constructor
that accept anything; it's an ellipsis function. If the comparison
between sizes evaluates to false, a decent compiler outputs an error
message such as "Error: Cannot convert
\texttt{ERROR\_Destination\_Type\_Too\_Narrow} to
\texttt{CompileTimeChecker <false>}.

\subsection{Partial Template Specialization}

\begin{verbatim}
template <class Window, class Controller>
class Widget{
  ... generic implementation ...
};

// Partial specialization of Widget
template <class Window>
class Widget<Window, MyController>{
  ... partially specialized implementation ...
};

template <class ButtonArg>
class Widget<Button<ButtonArg>, MyController>{
  ... further specialized implementation ...
};
\end{verbatim}

Unfortunately, partial template specialization does not apply to
functions—be they member or nonmember—which somewhat reduces the
flexibility and the granularity of what you can do:
\begin{itemize}
\item Although you can \textbf{totally specialize} member functions of
  a class template, you cannot \textbf{partially specialize} member
  functions.
\item You cannot partially specialize namespace-level (nonmember)
  template functions. The closest thing to partial specialization for
  namespace-level template functions is overloading (not for changing the
  return value or for internally used type).
\end{itemize}

\begin{verbatim}
template <class T, class U> T Fun(U obj); // primary template
template <class U> void Fun<void, U>(U obj); // illegal partial specialization
template <class T> T Fun (Window obj); // legal (overloading)
\end{verbatim}

\subsection{Local Classes}

\textbf{Local classes cannot define static member variables and cannot
  access nonstatic local variables}. What makes local classes truly
interesting is that you can use them in template functions. \textbf{Local
classes defined inside template functions can use the template
parameters of the enclosing function.}

\begin{verbatim}
class Interface{
public:
virtual void Fun() = 0;
};
template <class T, class P>
Interface* MakeAdapter(const T& obj, const P& arg){
  class Local : public Interface{
  public:
    Local(const T& obj, const P& arg): obj_(obj), arg_(arg) {}
    virtual void Fun(){
      obj_.Call(arg_);
    }
  private:
     T obj_;
     P arg_;
  };
  return new Local(obj, arg);
}
\end{verbatim}

It can be easily proven that any idiom that uses a local class can be
implemented using a template class outside the function. On the other
hand, local classes can simplify implementations and improve locality
of symbols.

Local classes do have a unique feature, though: They are
\textbf{final}. Outside users cannot derive from a class hidden in a
function. Without local classes, you'd have to add an unnamed
namespace in a separate translation unit.

\subsection{Mapping Integral Constants to Types}

\begin{verbatim}
template <int v>
struct Int2Type{
enum { value = v };
};
\end{verbatim}

\texttt{Int2Type} generates a distinct type for each distinct constant
integral value passed.  You can use \texttt{Int2Type} whenever you
need to "typify" an integral constant quickly. This way you can 
\textbf{select different functions, depending on the result of a
  compile-time calculation.} Effectively, you \textbf{achieve static
  dispatching on a constant integral value. }

For dispatching at runtime, you can use simple \texttt{if-else}
statements or the \texttt{switch} statement.  However, the
\texttt{if-else} statement requires both branches to compile
successfully, even when the condition tested by \texttt{if} is known
at compile time.

\begin{verbatim}
template <typename T, bool isPolymorphic>
class NiftyContainer{
  void DoSomething(){
    T* pSomeObj = ...;
    if (isPolymorphic){
      T* pNewObj = pSomeObj->Clone();
      ... polymorphic algorithm ...
    }
    else{
      T* pNewObj = new T(*pSomeObj);
      ... nonpolymorphic algorithm ...
    }
  }
};
\end{verbatim}

The polymorphic algorithm uses \texttt{pObj->Clone()},
\texttt{NiftyContainer::DoSomething }does not compile for any type
that doesn't define a member function \texttt{Clone()}.

If \texttt{T} has disabled its copy constructor (by making it
private), if \texttt{T} is a polymorphic type and the nonpolymorphic
code branch attempts \texttt{new T(*pObj)}, the code might fail to
compile.

\begin{verbatim}
template <typename T, bool isPolymorphic>
class NiftyContainer{
private:
  void DoSomething(T* pObj, Int2Type<true>){
    T* pNewObj = pObj->Clone();
    ... polymorphic algorithm ...
  }
  void DoSomething(T* pObj, Int2Type<false>){
    T* pNewObj = new T(*pObj);
    ... nonpolymorphic algorithm ...
  }
public:
  void DoSomething(T* pObj){
    DoSomething(pObj, Int2Type<isPolymorphic>());
  }
};
\end{verbatim}

\textbf{There is another solution, \texttt{if constexpr()}, the new
  feature provided by c++17.}

\subsection{Type-to-Type Mapping}

\begin{verbatim}
template <class T, class U>
T* Create(const U& arg){
  return new T(arg);
}
\end{verbatim}

If objects of type \texttt{Widget} are untouchable legacy code and must
take two arguments upon construction, the second being a fixed value
such as -1. How can you specialize Create so that it treats \texttt{Widget}
differently from all other types with a uniform interface?

\begin{verbatim}
// Illegal code — don't try this at home
template <class U>
Widget* Create<Widget, U>(const U& arg){
  return new Widget(arg, -1);
}

// rely on overloading
template <class T, class U>
T* Create(const U& arg, T /* dummy */){
  return new T(arg);
}
template <class U>
Widget* Create(const U& arg, Widget /* dummy */){
  return new Widget(arg, -1);
}
\end{verbatim}

Such a solution would incur the overhead of constructing an
arbitrarily complex object that remains unused.

\begin{verbatim}
template <typename T>
struct Type2Type{
  typedef T OriginalType;
};
template <class T, class U>
T* Create(const U& arg, Type2Type<T>){
  return new T(arg);
}
template <class U>
Widget* Create(const U& arg, Type2Type<Widget>){
  return new Widget(arg, -1);
}
// Use Create()
String* pStr = Create("Hello", Type2Type<String>());
Widget* pW = Create(100, Type2Type<Widget>());
\end{verbatim}

\subsection{Type Selection}

Sometimes generic code needs to select one type or another, depending
on a Boolean constant.

You might want to use an \texttt{std::vector} as your back-end
storage. Obviously, you cannot store polymorphic types by value, so
you must store pointers. On the other hand, you might want to store
nonpolymorphic types by value, because this is more efficient.

\begin{verbatim}
template <typename T, bool isPolymorphic>
struct NiftyContainerValueTraits{
  typedef T* ValueType;
};
template <typename T>
struct NiftyContainerValueTraits<T, false>{
  typedef T ValueType;
};
template <typename T, bool isPolymorphic>
class NiftyContainer{
  typedef NiftyContainerValueTraits<T, isPolymorphic> Traits;
  typedef typename Traits::ValueType ValueType;
};
\end{verbatim}

This way of doing things is unnecessarily clumsy. Moreover, it doesn't
scale: For each type selection, you must define a new traits class
template.

\begin{verbatim}
template <bool flag, typename T, typename U>
struct Select{
  typedef T Result;
};
template <typename T, typename U>
struct Select<false, T, U>{
  typedef U Result;
};

template <typename T, bool isPolymorphic>
class NiftyContainer{
  typedef typename Select<isPolymorphic, T*, T>::Result ValueType;
}
\end{verbatim}

\subsection{Detecting Convertibility and Inheritance at Compile Time}

 In a generic function, you can rely on an optimized algorithm if a
 class implements a certain interface. Discovering this at compile
 time means not having to use \texttt{dynamic\_cast}, which is costly
 at runtime.

 Detecting inheritance relies on a more general mechanism, that of
 detecting convertibility. The more general problem is, How can you
 detect whether an arbitrary type \texttt{T} supports automatic
 conversion to an arbitrary type \texttt{U}?

 There is a surprising amount of power in \texttt{sizeof}: You can
 apply \texttt{sizeof} to any expression, no matter how complex, and
 \texttt{sizeof} \textbf{returns its size without actually evaluating
   that expression at runtime. }

 The idea of conversion detection relies on using \texttt{sizeof} in
 conjunction with overloaded functions. We provide two overloads of a
 function: \textbf{One accepts the type to convert to (\texttt{U}), and the
 other accepts just about anything else. } If the function that
accepts a \texttt{U} gets called, we know that \texttt{T} is
convertible to \texttt{U}.

\begin{verbatim}
typedef char Small;
class Big { char dummy[2]; };
Small Test(U);
Big Test(...);
const bool convExists = sizeof(Test(T())) == sizeof(Small);
\end{verbatim}
Passing a C++ object to a function with ellipses has undefined
results, but this doesn't matter. Nothing actually calls the
function. It's not even implemented. Recall that \texttt{sizeof} does
not evaluate its argument.

There is one little problem. If \texttt{T} makes its default constructor
private, the expression \texttt{T()} fails to compile. Fortunately,
there is a simple solution, just use a strawman function
returning a \texttt{T}.  \texttt{MakeT} and \texttt{Test} not only
don't do anything but don't even really exist at all.

\begin{verbatim}
template <class T, class U>
class Conversion{
  typedef char Small;
  class Big { char dummy[2]; };
  static Small Test(U);
  static Big Test(...);
  static T MakeT(); // not implemented
public:
  enum { exists = sizeof(Test(MakeT())) == sizeof(Small) };
};
cout << Conversion<size_t, vector<int> >::exists << ' ';
// return 0, because that constructor is explicit.
\end{verbatim}

We can implement one more constant inside
\texttt{Conversion::sameType}, which is true if \texttt{T} and
\texttt{U} represent the same type:

\begin{verbatim}
template <class T, class U>
class Conversion{
  ... as above ...
  enum { sameType = false };
};
template <class T>
class Conversion<T, T>{
public:
  enum { exists = 1, sameType = 1 };
};
#define SUPERSUBCLASS(T, U) \
(Conversion<const U*, const T*>::exists && \
!Conversion<const T*, const void*>::sameType)
\end{verbatim}

There are only three cases in which \texttt{const U*} converts
implicitly to \texttt{const T*}: 
\begin{enumerate}
\item \texttt{T} is the same type as \texttt{U}
\item \texttt{T} is an unambiguous public base of \texttt{U}
\item \texttt{T} is \texttt{void}.
\end{enumerate}

Using \texttt{const} in \texttt{SUPERSUBCLASS}, we're always on the
safe side, we don't want the conversion test to fail due to
\texttt{const} issues.

Why use \texttt{SUPERSUBCLASS} and not the cuter \texttt{BASE\_OF} or
\texttt{INHERITS}? Think with \texttt{INHERITS(T, U)} it was a
constant struggle to say which way the test worked.

\subsection{A Wrapper Around \texttt{type\_info} }

tandard C++ provides the \texttt{std::type\_info} class, which gives you the
ability to investigate object types at runtime. You typically use
\texttt{type\_info} in conjunction with the \texttt{typeid}
operator. The \texttt{typeid} operator returns a reference to a
\texttt{type\_info} object:

\begin{verbatim}
void Fun(Base* pObj){
  // Compare the two type_info objects corresponding to the type of *pObj and Derived
  if (typeid(*pObj) == typeid(Derived)){
    ... aha, pObj actually points to a Derived object ...
  }
}
\end{verbatim}

In addition to supporting the comparison operators \texttt{operator==}
and \texttt{operator!=}, \texttt{type\_info} provides two more
functions:

\begin{enumerate}
\item The \texttt{name} member function returns a textual
  representation of a type, in the form of \texttt{const char*}.
\item he before member function introduces an implementation's
  collation ordering relationship for \texttt{type\_info} objects.
\item The \texttt{type\_info} class disables the copy constructor and
  assignment operator, which makes storing \texttt{type\_info} objects
  impossible.
\item The objects returned by \texttt{typeid} have static storage, so
  you don't have to worry about lifetime issues.  
\end{enumerate}

You do have to worry about pointer identity, the standard does not
guarantee that each invocation returns a reference to the same
\texttt{type\_info} object. Consequently, you cannot compare pointers
to \texttt{type\_info} objects. What you should do is to store pointers
to \texttt{type\_info} objects and compare them by applying
\texttt{type\_info::operator==} to the dereferenced pointers.

If you want to use STL's ordered containers with \texttt{type\_info},
you must write a little functor and deal with pointers. All this is
clumsy enough to mandate a wrapper class around \texttt{type\_info}
that stores a pointer to a \texttt{type\_info} object and provides:
\begin{enumerate}
\item All member functions of \texttt{type\_info}
\item Value semantics (public copy constructor and assignment
  operator)
\item Seamless comparisons by defining \texttt{operator<} and
  \texttt{operator==}
\end{enumerate}

\begin{verbatim}
class TypeInfo{
public:
  // Constructors/destructors
  TypeInfo(); // needed for containers
  TypeInfo(const std::type_info&);
  TypeInfo(const TypeInfo&);
  TypeInfo& operator=(const TypeInfo&);
  // Compatibility functions
  bool before(const TypeInfo&) const;
  const char* name() const;
private:
  const std::type_info* pInfo_;
};
// Comparison operators
bool operator==(const TypeInfo&, const TypeInfo&);
bool operator!=(const TypeInfo&, const TypeInfo&);
bool operator<(const TypeInfo&, const TypeInfo&);
bool operator<=(const TypeInfo&, const TypeInfo&);
bool operator>(const TypeInfo&, const TypeInfo&);
bool operator>=(const TypeInfo&, const TypeInfo&);

void Fun(Base* pObj){
  TypeInfo info = typeid(Derived);
  if (typeid(*pObj) == info){
    ... pBase actually points to a Derived object ...
  }
}
\end{verbatim}

\textbf{The cloning factory in Chapter 8 and one double-dispatch
  engine in Chapter 11 put \texttt{TypeInfo} to good use. }

\subsection{\texttt{NullType} and \texttt{EmptyType}}

\begin{verbatim}
class NullType {};
struct EmptyType {};
\end{verbatim}

You can use \texttt{NullType} for cases in which a type must be there
syntactically but doesn't have a semantic sense. You can use
\texttt{EmptyType} as a default ("don't care") type for a template.

\subsection{Type Traits}

\textbf{Traits are a generic programming technique that allows
  compile-time decisions to be made based on types, much as you would
  make runtime decisions based on values.}

\subsubsection{ Implementing Pointer Traits}

\begin{verbatim}
template <typename T>
class TypeTraits{
private:
  template <class U> 
  struct PointerTraits{
    enum { result = false };
    typedef NullType PointeeType;
  };
  template <class U> 
  struct PointerTraits<U*>{
    enum { result = true };
    typedef U PointeeType;
  };
public:
  enum { isPointer = PointerTraits<T>::result };
  typedef PointerTraits<T>::PointeeType PointeeType;
};

const bool iterIsPtr = TypeTraits<vector<int>::iterator>::isPointer;
cout << "vector<int>::iterator is " << iterIsPtr ? "fast" : "smart" << '\n';
\end{verbatim}

Similarly, TypeTraits implements an \texttt{isReference} constant and
a \texttt{ReferencedType} type definition.

Detection of pointers to members is a bit different. The
specialization needed is as follows:
\begin{verbatim}
template <typename T>
class TypeTraits{
private:
  template <class U> 
  struct PToMTraits{
    enum { result = false };
  };
template <class U, class V>
  struct PToMTraits<U V::*>{
    enum { result = true };
  };
public:
  enum { isMemberPointer = PToMTraits<T>::result };
};
\end{verbatim}

\subsubsection{Detection of Fundamental Types}

\texttt{TypeTraits<T>} implements an \texttt{isStdFundamental}
compile-time constant that says whether or not \texttt{T} is a
standard fundamental type.

In Section 3, we will know an \texttt{TypeList} and the expression
\begin{verbatim}
TL::IndexOf<T, TYPELIST_nn(comma-separated list of types)>::value
\end{verbatim}
returns the zero-based position of \texttt{T} in the list, or –1 if
\texttt{T} does not figure in the list.

\begin{verbatim}
template <typename T>
class TypeTraits
{
... as above ...
public:
  typedef TYPELIST_4(unsigned char, unsigned short int, unsigned int, unsigned long int) UnsignedInts;
  typedef TYPELIST_4(signed char, short int, int, long int) SignedInts;
  typedef TYPELIST_3(bool, char, wchar_t) OtherInts;
  typedef TYPELIST_3(float, double, long double) Floats;
  enum { isStdUnsignedInt = TL::IndexOf<T, UnsignedInts>::value >= 0 };
  enum { isStdSignedInt = TL::IndexOf<T, SignedInts>::value >= 0 };
  enum { isStdIntegral = isStdUnsignedInt || isStdSignedInt || TL::IndexOf <T, OtherInts>::value >= 0 };
  enum { isStdFloat = TL::IndexOf<T, Floats>::value >= 0 };
  enum { isStdArith = isStdIntegral || isStdFloat };
  enum { isStdFundamental = isStdArith || isStdFloat || Conversion<T, void>::sameType };
  ...
};
\end{verbatim}

\subsubsection{Optimized Parameter Types}

Given an arbitrary type \texttt{T}, what is the most efficient way of
passing and accepting objects of type \texttt{T} as arguments to
functions? In general, the most efficient way is to pass elaborate
types by reference and scalar types by value.

A detail that must be carefully handled is that C++ does not allow
references to references. Thus, if \texttt{T} is already a reference,
you should not add one more reference to it.

\begin{verbatim}
template <typename T>
class TypeTraits{
  ... as above ...
public:
  typedef Select<isStdArith || isPointer || isMemberPointer, T,ReferencedType&>::Result ParameterType;
};
\end{verbatim}

\subsubsection{Stripping Qualifiers}

\begin{verbatim}
template <typename T>
class TypeTraits{
  ... as above ...
private:
  template <class U> struct UnConst{
    typedef U Result;
  };
  template <class U> struct UnConst<const U>{
    typedef U Result;
  };
public:
  typedef UnConst<T>::Result NonConstType;
};
\end{verbatim}

\subsubsection{Using \texttt{TypeTraits}}

\begin{verbatim}
enum CopyAlgoSelector { Conservative, Fast };
// Conservative routine-works for any type
template <typename InIt, typename OutIt>
OutIt CopyImpl(InIt first, InIt last, OutIt result, Int2Type<Conservative>){
  for (; first != last; ++first, ++result)
  *result = *first;
  return result;
}
// Fast routine-works only for pointers to raw data
template <typename InIt, typename OutIt>
OutIt CopyImpl(InIt first, InIt last, OutIt result, Int2Type<Fast>){
  const size_t n = last-first;
  BitBlast(first, result, n * sizeof(*first));
  return result + n;
}
template <typename InIt, typename OutIt>
OutIt Copy(InIt first, InIt last, OutIt result){
  typedef TypeTraits<InIt>::PointeeType SrcPointee;
  typedef TypeTraits<OutIt>::PointeeType DestPointee;
  enum { copyAlgo = 
         TypeTraits<InIt>::isPointer &&
         TypeTraits<OutIt>::isPointer &&
         TypeTraits<SrcPointee>::isStdFundamental &&
         TypeTraits<DestPointee>::isStdFundamental &&
         sizeof(SrcPointee) == sizeof(DestPointee) ? Fast : Conservative };
  return CopyImpl(first, last, result, Int2Type<copyAlgo>);
}
\end{verbatim}

The drawback of Copy is that it doesn't accelerate everything that
could be accelerated. For example, you  might have a plain C-like
struct containing nothing but primitive data—a so-called plain old
data, or POD, structure.
\begin{verbatim}
template <typename T> 
struct SupportsBitwiseCopy{
  enum { result = TypeTraits<T>::isStdFundamental };
};
template<> 
struct SupportsBitwiseCopy<MyType>{
  enum { result = true };
};
template <typename InIt, typename OutIt>
OutIt Copy(InIt first, InIt last, OutIt result, Int2Type<true>){
  typedef TypeTraits<InIt>::PointeeType SrcPointee;
  typedef TypeTraits<OutIt>::PointeeType DestPointee;
  enum { useBitBlast =
         TypeTraits<InIt>::isPointer &&
         TypeTraits<OutIt>::isPointer &&
         SupportsBitwiseCopy<SrcPointee>::result &&
         SupportsBitwiseCopy<DestPointee>::result &&
         sizeof(SrcPointee) == sizeof(DestPointee) };
  return CopyImpl(first, last, Int2Type<useBitBlast>);
}
\end{verbatim}

\subsubsection{Summary}

The most important point is that the compiler always find the best
match of template specialization.
%%% Local Variables:
%%% mode: latex
%%% TeX-master: "../DesignPattern"
%%% End:
