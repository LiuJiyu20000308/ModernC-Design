\newpage
\section{Small-Object Allocation}

\subsection{Why we need smallObj allocation?}

For various reasons, polymorphic behavior being the most important,
these small objects cannot be stored on the stack and must live on the
free store . C++ provides the operators \texttt{new} and
\texttt{delete} as the primary means of using the free store.
However, these operators are general purpose and perform badly for
allocating small objects.

For occult reasons, the default allocator (\texttt{malloc, realloc, free}) is
notoriously slow. In addition to being slow, the genericity of the
default C++ allocator makes it very space inefficient for small
objects.  Usually, the bookkeeping memory amounts to a few extra bytes
(4 to 32) for each block allocated with \texttt{new}, If you allocate
8-byte objects, the per-object overhead becomes 50\% to 400\%.

\subsection{The workings of a memory allocator}

\begin{verbatim}
struct MemControlBlock{
  std::size_t size_;
  bool available_;
};
\end{verbatim}

For each allocation request, a linear search of memory blocks finds a
suitable block for the requested size. Each deallocation incurs,
again, a linear search for figuring out the memory block that precedes
the block  being deallocated, and an adjustment of its size.

\begin{verbatim}
struct MemControlBlock{
  bool available_ ;
  MemControlBlock* prev_;
  MemControlBlock* next_;
};
\end{verbatim}

If you store pointers to the previous and next
\texttt{MemControlBlock} in each \texttt{MemControlBlock}, you can
achieve constant-time deallocation.

\subsection{A Small-Object Allocator}

The small-object allocator described in this chapter sports a
four-layered structure:
\begin{enumerate}
\item \texttt{Chunk} contains and manages a chunk of memory consisting
  of an integral number of fixedsize blocks. \texttt{Chunk} contains
  logic that allows you to allocate and deallocate memory blocks
\item A \texttt{FixedAllocator} object uses \texttt{Chunk} as a
  building block. \texttt{FixedAllocator}'s primary purpose is to
  satisfy memory requests that go beyond a \texttt{Chunk}'s
  capacity. \texttt{FixedAllocator} does this by aggregating an array
  of \texttt{Chunk}s.
\item \texttt{SmallObjAllocator} provides general allocation and
  deallocation functions. A \texttt{SmallObjAllocator} holds several
  \texttt{FixedAllocator} objects, each specialized for allocating
  objects of one size.
\item Finally, \texttt{SmallObject} wraps \texttt{FixedAllocator} to
  offer encapsulated allocation services for C++ classes. SmallObject
  overloads operator \texttt{new} and operator \texttt{delete} and
  passes them to a \texttt{SmallObjAllocator} object. 
\end{enumerate}

\subsection{Chunk}

\begin{verbatim}
struct Chunk{
  void Init(std::size_t blockSize, unsigned char blocks);
  void* Allocate(std::size_t blockSize);
  void Deallocate(void* p, std::size_t blockSize);
  void Reset(std::size_t blockSize, unsigned char blocks);
  void Release();
  unsigned char* pData_;
  unsigned char firstAvailableBlock_, blocksAvailable_;
};
\end{verbatim}

\texttt{firstAvailableBlock\_}  holds the index of the first block
available in this chunk, \texttt{blocksAvailable\_} holds the number of
blocks available in this chunk.

\texttt{Chunk} does not define constructors, destructors, or assignment
operator. Defining proper copy semantics at this level hurts
efficiency at upper level. Allocating and deallocating a
block inside a Chunk takes constant time.

Why we use \texttt{unsinged char} but not \texttt{unsigned short} (2
bytes on many machines):
\begin{enumerate}
\item We cannot allocate blocks smaller than \texttt{sizeof(unsigned
    short)}, which is awkward because we're building a small-object
  allocator.
\item Imagine you build an allocator for 5-byte blocks. In this case,
casting a pointer that points to such a 5-byte block to unsigned int
engenders undefined behavior.
\end{enumerate}

\subsection{FixedAllocator}

\begin{verbatim}
class FixedAllocator{
private:
    // Internal functions        
    void DoDeallocate(void* p);
    Chunk* VicinityFind(void* p);

    std::size_t blockSize_;
    unsigned char numBlocks_;
    typedef std::vector<Chunk> Chunks;
    Chunks chunks_;
    Chunk* allocChunk_;
    Chunk* deallocChunk_;

    // For ensuring proper copy semantics
    mutable const FixedAllocator* prev_;
    mutable const FixedAllocator* next_;
public:
    explicit FixedAllocator(std::size_t blockSize = 0);
    FixedAllocator(const FixedAllocator&);
    FixedAllocator& operator=(const FixedAllocator&);
    ~FixedAllocator();
    void Swap(FixedAllocator& rhs);

    // Allocate a memory block
    void* Allocate();
    void Deallocate(void* p);
    std::size_t BlockSize() const{ return blockSize_; }
};
\end{verbatim}

\texttt{allocChunk\_} holds a pointer to the last chunk that was used
for an allocation. Whenever an allocation request comes,
\texttt{FixedAllocator::Allocate} first checks \texttt{allocChunk\_}
for available space. If not, a linear search occurs.

 \texttt{deallocChunk\_} points to the last \texttt{Chunk} object that
 was used for a deallocation. Whenever a deallocation occurs,
 \texttt{deallocChunk\_} is checked first. Then, if it's the wrong
 chunk, \texttt{Deallocate} performs a linear search:
 \begin{enumerate}
 \item during deallocation, a chunk is freed only when there are two
empty chunks.
\item \texttt{chunks\_} is searched starting from
  \texttt{deallocChunk\_} and going up and down with two iterators.
\end{enumerate}

\subsection{SmallObjAllocator}
\begin{verbatim}
class SmallObjAllocator{
public:
  SmallObjAllocator(std::size_t chunkSize,std::size_t maxObjectSize);
  void* Allocate(std::size_t numBytes);
  void Deallocate(void* p, std::size_t size);
private:
  std::vector<FixedAllocator> pool_;
  FixedAllocator* pLastAlloc_;
  FixedAllocator* pLastDealloc_;
  std::size_t chunkSize_;
  std::size_t maxObjectSize_;
};
\end{verbatim}

The \texttt{chunkSize} parameter is the default chunk size (the length
in bytes of each Chunk object), and \texttt{maxObjectSize} is the
maximum size of objects that must be considered to be "small."
\texttt{SmallObjAllocator} forwards requests 
for blocks larger than \texttt{maxObjectSize} directly to
\texttt{::operator new}.

We store \texttt{FixedAllocators }only for sizes that are requested at
least once. This way \texttt{pool\_} can accommodate
various object sizes without growing too much. To improve lookup
speed, \texttt{pool\_} is kept sorted by block.
size.

When an allocation request arrives, \texttt{pLastAlloc\_} is checked first. If
it is not of the correct size, \texttt{SmallObjAllocator::Allocate}
performs a binary search in \texttt{pool\_}. Deal location requests are 
handled in a similar way.

\subsection{Small Object}

\begin{verbatim}
class SmallObject{
public:
  static void* operator new(std::size_t size);
  static void operator delete(void* p, std::size_t size);
  virtual ~SmallObject() {}
};
\end{verbatim}

 In standard C++ you can overload the default operator delete in two
 ways—either as
\begin{verbatim}
void operator delete(void* p);
\end{verbatim}
or as
\begin{verbatim}
void operator delete(void* p, std::size_t size);
\end{verbatim}

 To avoid the overhead of storing the size of the actual object to
 which p points, the compiler does a hat trick: It generates code that
 figures out the size on the fly. Four possible techniques of
 achieving that are listed here:
 \begin{enumerate}
 \item Pass a Boolean flag to the destructor meaning "Call/don't call
   operator delete after destroying the object." Base's destructor is
   virtual, so, \texttt{delete p} will reach the right object,
   \texttt{Derived}. At that time, the size of the object is known
   statically—it's \texttt{sizeof(Derived)}, and the compiler simply
   passes this constant to operator \texttt{delete}.
 \item You can arrange that each destructor, after destroying the
   object, returns \texttt{sizeof(Class)}.
 \item Implement a hidden virtual member function that gets the size
   of an object, say \texttt{\_Size()}.  
 \item Store the size directly somewhere in the virtual function table
   (vtable) of each class. This solution is both flexible and
   efficient, but less easy to implement.
 \end{enumerate}

 We need a unique \texttt{SmallObjAllocator} object for the whole
 application. That \texttt{SmallObjAllocator} must be properly
 constructed and properly destroyed, which is a thorny issue on its
 own. we solve this problem thoroughly with its
 \texttt{SingletonHolder} template.

\begin{verbatim}
typedef Singleton<SmallObjAllocator> MyAlloc;
void* SmallObject::operator new(std::size_t size){
  return MyAlloc::Instance().Allocate(size);
}
void SmallObject::operator delete(void* p, std::size_t size){
  MyAlloc::Instance().Deallocate(p, size);
}
\end{verbatim}

 \subsection{Multithreading issues}

The unique \texttt{SmallObjAllocator} is shared by all instances of
\texttt{SmallObject}. If these instances belong to different threads,
we end up sharing the \texttt{SmallObjAllocator} between multiple
threads.

\begin{verbatim}
template <template <class T> class ThreadingModel>
class SmallObject : public ThreadingModel<SmallObject>{
  ... as before ...
}

template <template <class T> class ThreadingModel>
void* SmallObject<ThreadingModel>::operator new(std::size_t size){
  Lock lock;
  return MyAlloc::Instance().Allocate(size);
}
template <template <class T> class ThreadingModel>
void SmallObject<ThreadingModel>::operator delete(void* p, std::size_t size){
  Lock lock;
  MyAlloc::Instance().Deallocate(p, size);
}
\end{verbatim}
%%% Local Variables:
%%% mode: latex
%%% TeX-master: "../DesignPattern"
%%% End:
