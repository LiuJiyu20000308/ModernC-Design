\section{Policy-Based Class Design}

In brief, policy-based class design
fosters assembling a class with complex behavior out of many little classes 
(called policies), each of which
takes care of only one behavioral or structural aspect.

The generic SingletonHolder class \textbf{template} (Chapter
6) uses policies for managing lifetime and thread safety. {SmartPtr} (Chapter 7) 
is built almost entirely from policies. The \textbf{double-dispatch engine} in Chapter 11 
uses policies for selecting various trade-offs. The
generic \textbf{Abstract Factory} implementation in Chapter 9 uses a policy for choosing a
creation method.

\subsection{Failure of the Do-it-all Interface}

Implementing everything under the umbrella of a do-it-all interface is not a good solution, for several
reasons:

\begin{itemize}
    \item Intellectual overhead, sheer size, and inefficiency.
    \item Loss of static type safety. A design should enforce most constraints at compile time. (No two singleton objects)
\end{itemize}

\subsection{Multiple Inheritance to the Rescue?}
For example, the user would build a multi-threaded, reference-counted smart pointer
class by inheriting some 
\texttt{BaseSmartPtr} class and two classes: \texttt{MultiThreaded}
and  \texttt{RefCounted}. Any experienced class 
designer knows that such a naive design does not work.

The problems with assembling separate features by using
multiple inheritance are as follows:
\begin{itemize}
\item \textbf{Mechanics}. There is no boilerplate code to assemble the
  inherited components in a controlled manner. \textbf{The language applies
  simple superposition in combining the base classes and establishes a
  set of simple rules for accessing their members.  }
\item \textbf{Type information}. The base classes do not have enough type
  information to carry out their tasks. 
\item \textbf{State manipulation}. Various behavioral aspects implemented with
  base classes must manipulate the same state. This means that they
  must use virtual inheritance to inherit a base class that holds the
  state. This complicates the design and makes it more rigid because
  the premise was that user classes inherit library classes, not vice versa.
\end{itemize}

\subsection{Templates}

Benefits:
\begin{itemize}
\item Class templates are customizable in ways not supported by
  regular classes.
\item for class templates with multiple parameters, you can use
  partial template specialization.
\end{itemize}

As soon as you try to implement such designs, you stumble upon several
problems that are not self-evident:
\begin{itemize}
\item You cannot specialize structure.
\item Specialization of member functions does not scale: you cannot
  specialize individual member functions for templates with multiple
  template parameters.
\item The library writer cannot provide multiple \textbf{default} values.
\end{itemize}

Multiple inheritance and templates foster complementary trade-offs:
\begin{itemize}
\item Multiple inheritance has scarce mechanics; templates have rich
mechanics
\item Multiple inheritance loses type information, which abounds
in templates.
\item Specialization of templates does not scale, but multiple
inheritance scales quite nicely.
\item You can provide only one default for
a template member function, but you can write an unbounded 
number of base classes.
\end{itemize}

\subsection{Policy Classes}

A \textbf{policy} defines a class interface or a class template interface. The
interface consists of one or all of the following: \textbf{inner type
definitions, member functions, and member variables}. The
implementations of a policy are called \textbf{policy classes}. Policy classes
are not intended for stand-alone use; instead, they are inherited by,
or contained within, other classes.

\begin{verbatim}
template <class T>
struct OpNewCreator{
  static T* Create(){
    return new T;
  }
};

template <class T>
struct MallocCreator{
  static T* Create(){
    void* buf = std::malloc(sizeof(T));
    if (!buf) return 0;
    return new(buf) T;
  }
};

template <class T>
struct PrototypeCreator{
  PrototypeCreator(T* pObj = 0):pPrototype_(pObj){}
  T* Create(){
    return pPrototype_ ? pPrototype_->Clone() : 0;
  }
  T* GetPrototype() { return pPrototype_; }
  void SetPrototype(T* pObj) { pPrototype_ = pObj; }
private:
  T* pPrototype_;
};
\end{verbatim}

The classes that use one or more policies are called hosts or
\textbf{host classes}.

\begin{verbatim}
template <class CreationPolicy>
class WidgetManager : public CreationPolicy{
  ...
};
typedef WidgetManager< OpNewCreator<Widget> > MyWidgetMgr;
\end{verbatim}

\textbf{It is the user of \texttt{WidgetManager} who chooses the creation
  policy}. This is the gist of policy-based class design.

\subsection{Implementing Policy Classes with Template Template
  Parameters}

The policy's template argument is redundant. In this case, we can use
\textbf{template template parameters} for specifying policies, as shown in the
following:
\begin{verbatim}
template <template <class> class CreationPolicy = OpNewcreator>
class WidgetManager : public CreationPolicy<Widget>{
  ...
};
typedef WidgetManager<OpNewCreator> MyWidgetMgr;
\end{verbatim}

Using template template parameters with policy classes is not simply a
matter of convenience; sometimes, it is essential that the host class
have access to the template so that the host can instantiate it with a
different type. For example:

\begin{verbatim}
template <template <class> class CreationPolicy = OpNewcreator>
class WidgetManager : public CreationPolicy<Widget>{
  void DoSomething(){
    Gadget* pW = CreationPolicy<Gadget>().Create();
  }
};
\end{verbatim}

Benefits of using policies:
\begin{itemize}
\item you can change policies from theoutside as easily as changing a
  template argument when you instantiate \texttt{WidgetManager}.
\item you can provide your own policies that are specific to your
  concrete application.
\item Policies allow you to generate designs by combining simple
  choices in a typesafe manner.
\item the binding between a host class and its policies is done at
  compile time, the code is tight and efficient, comparable to its
  handcrafted equivalent.
\end{itemize}

\subsection{Destructors of Policy Classes}

The user can automatically convert a host class to a policy and later
\texttt{delete} that pointer. Unless the policy class defines a virtual
destructor, applying delete to a pointer to the policy class has
undefined behavior.

Defining a virtual destructor for a policy, however, works against its
static nature and hurts performance. The lightweight, effective
solution that policies should use is to define a nonvirtual protected
destructor:
\begin{verbatim}
template <class T>
struct OpNewCreator{
protected:
  ~OpNewCreator() {}
};
\end{verbatim}

Because the destructor is protected, \textbf{only derived classes can destroy
policy objects}, so it's impossible for outsiders to apply delete to a
pointer to a policy class.

\subsection{Enriched Policies}

The \texttt{Creator} policy prescribes only one member function,
\texttt{Create}. However, \texttt{PrototypeCreator} defines two more functions:
\texttt{GetPrototype} and \texttt{SetPrototype}.

A user who uses a prototype-based Creator policy class can write the
following code: 
\begin{verbatim}
typedef WidgetManager<PrototypeCreator> MyWidgetManager;

Widget* pPrototype = ...;
MyWidgetManager mgr;
mgr.SetPrototype(pPrototype);
\end{verbatim}

If the user later decides to use a creation policy that does not
support prototypes, \textbf{the compiler pinpoints the spots where the
prototype-specific interface was used.} This is exactly what should be
expected from a sound design.

\subsection{Optional Functionality Through Incomplete Instantiation}

If a member function of a class template is never used, \textbf{it is not even
instantiated—the compiler does not look at it at all}, except perhaps
for syntax checking.

\begin{verbatim}
template <template <class> class CreationPolicy>
class WidgetManager : public CreationPolicy<Widget>{
  void SwitchPrototype(Widget* pNewPrototype){
    CreationPolicy<Widget>& myPolicy = *this;
    delete myPolicy.GetPrototype();
    myPolicy.SetPrototype(pNewPrototype);
  }
};
\end{verbatim}
The resulting context is very interesting:
\begin{itemize}
\item If the user instantiates \texttt{WidgetManager} with a Creator
  policy class that does not support prototypes and tries to use
  \texttt{SwitchPrototype}, a compile-time error occurs.
\item  If the user instantiates \texttt{WidgetManager} with a Creator
  policy class that does not support prototypes and does not try to
  use \texttt{SwitchPrototype}, the program is valid.
\end{itemize}
This all means that \texttt{WidgetManager} can benefit from optional
enriched interfaces but still work correctly with poorer interfaces.

\subsection{Compatible and Incompatible Policies}

Suppose you create two instantiations of \texttt{SmartPtr }:
\texttt{FastWidgetPtr}, a pointer with out checking, and 
\texttt{SafeWidgetPtr}, a pointer with checking before dereference.
It is natural to accept the conversion from \texttt{FastWidgetPtr} to
\texttt{SafeWidgetPtr}, but freely converting \texttt{SafeWidgetPtr}
objects to \texttt{FastWidgetPtr} objects is dangerous. 

The best, most scalable way to implement conversions between policies
is to initialize and copy \texttt{SmartPtr} objects policy by policy,
as shown below:

\begin{verbatim}
template<class T,template <class> class CheckingPolicy>
class SmartPtr : public CheckingPolicy<T>{
  template<class T1,template <class> class CP1,>
  SmartPtr(const SmartPtr<T1, CP1>& other)
    : pointee_(other.pointee_), CheckingPolicy<T>(other){ ... }
};
\end{verbatim}

When you initialize a \texttt{SmartPtr<Widget, EnforceNotNull> } with a
\texttt{SmartPtr<ExtendedWidget, NoChecking>}. The compiler tries to
match \texttt{SmartPtr<ExtendedWidget, NoChecking>} to
\texttt{EnforceNotNull}'s constructors.

If \texttt{EnforceNotNull}
implements a \textbf{constructor} that accepts a \texttt{NoChecking} object,
then the compiler matches that constructor. If \texttt{NoChecking}
implements a \textbf{conversion} operator to \texttt{EnforceNotNull}, that 
conversion is invoked. In any other case, the code fails to compile.

Although conversions from \texttt{NoChecking} to
\texttt{EnforceNotNull} and even vice versa are quite sensible, some
conversions don't make any sense at all.  As soon as you try to
confine a pointer to another ownership policy, you break the invariant
that makes reference counting work.

In conclusion, conversions that change the ownership policy should not
be allowed implicitly and should be treated with maximum care.

\subsection{ Decomposing a Class into Policies}

Two policies that do not interact with each other are orthogonal. By
this definition, the Array and the Destroy policies are not
orthogonal.

Nonorthogonal policies are an imperfection you should strive to
avoid. \textbf{They reduce compile-time type safety and complicate the design
  of both the host class and the policy classes.}

If you must use nonorthogonal policies, you can minimize dependencies
by passing a policy class as an argument to another policy class's
template function. However, this decreases encapsulation.
%%% Local Variables:
%%% mode: latex
%%% TeX-master: "../DesignPattern.tex"
%%% End:
