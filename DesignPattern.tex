\documentclass[a4paper,twoside]{ctexart}
\usepackage{geometry}
\geometry{margin=1cm,vmargin={0pt,1cm}}
\setlength{\topmargin}{-2cm}
\setlength{\paperheight}{23cm}
\setlength{\paperwidth}{18cm}
\setlength{\textheight}{19.6cm}
\setlength{\textwidth}{15cm}
\usepackage{makecell}
%\usepackage{fancyhdr}
\usepackage{siunitx}
\usepackage{amssymb}
\usepackage{indentfirst}
\setlength{\parindent}{0.5em}

\pagenumbering{arabic} 

% useful packages.
\usepackage{multirow}
\usepackage{caption}
\usepackage{mathrsfs}
\usepackage{amsfonts}
\usepackage{amsmath}
\usepackage{amsthm}
\usepackage{enumerate}
\usepackage{xcolor,graphicx,float,subfigure}
\usepackage{epstopdf}
\usepackage{multicol}
\usepackage{fancyhdr}
\usepackage{layout}
\usepackage{listings}
\usepackage{dsfont}
\lstset{language=Matlab}
\lstset{breaklines}
\lstset{extendedchars=false}
\usepackage[colorlinks,linkcolor=blue]{hyperref}
\usepackage{xcolor}
%\usepackage{cite}
%\usepackage[numbers,sort&compress]{natbib} 
%\setcitestyle{open={},close={}}
%\usepackage{natbibspacing}
%\renewcommand{\refname}{}
\usepackage{anyfontsize}

\usepackage{tikz}
\usetikzlibrary{calc}
\usetikzlibrary{arrows.meta}
\tikzset{
  dot/.style={
    circle, fill=black, inner sep=1pt, outer sep=0pt
  },
  dot label/.style={
    circle, inner sep=0pt, outer sep=1pt
  }
  arrow1/.style = {
    draw = black, thick, -{Latex[length = 4mm, width = 1.5mm]},
  }
}

\newtheorem{theorem}{定理}[section]
\newtheorem{corollary}[theorem]{推论}
\newtheorem{lemma}[theorem]{引理}
\newtheorem{definition}[theorem]{定义}
\newtheorem{proposition}[theorem]{性质}
\newtheorem{example}[theorem]{例子}
\newtheorem{notation}[theorem]{记号}
\newtheorem{algorithm}[theorem]{算法}


\newcommand{\dif}{\mathrm{d}}
\newcommand{\avg}[1]{\left\langle #1 \right\rangle}
\newcommand{\difFrac}[2]{\frac{\dif #1}{\dif #2}}
\newcommand{\pdfFrac}[2]{\frac{\partial #1}{\partial #2}}
\newcommand{\OFL}{\mathrm{OFL}}
\newcommand{\UFL}{\mathrm{UFL}}
\newcommand{\fl}{\mathrm{fl}}
\newcommand{\op}{\odot}
\newcommand{\Eabs}{E_{\mathrm{abs}}}
\newcommand{\Erel}{E_{\mathrm{rel}}}

\newcommand{\Zero}{\hat{0}}
\newcommand{\One}{\hat{1}}
\newcommand{\Int}{\mathrm{int}}
\newcommand{\unitV}{\mathds{1}}

\newcommand{\bmi}{\mathbf{i}}
\newcommand{\bmj}{\mathbf{j}}
\newcommand{\bmn}{\mathbf{n}}

\newcommand{\dist}[2]{\text{dist}\left(#1, #2\right)}
\newcommand{\scientific}[2]{#1 \times 10^{#2}}


%\newcommand{\Dim}{{\mathbf{D}}}
\newcommand{\Dim}{{\scriptsize \textsf{D}}}
\newcommand{\me}{\mathrm{e}}
\newcommand{\mi}{\mathrm{i}}

%\newcommand{\mod}{\mathrm{mod}}
\newcommand{\curve}[1]{\widetilde{#1}}
%\newcommand{\dt}{\delta t}
\newcommand{\dt}{\tau}
\newcommand{\isCovered}{\mathbin{ < \! \! \! \! \cdot }}
%\newcommand{\cIncluded}{\mathbin{ \prec \! \! \! \cdot }}
\newcommand{\coveredBy}{\lhd}
%\newcommand{\regrz}[1]{\mathrm{cl}\left(\mathrm{int}\left(#1\right)\right)}
\newcommand{\regrz}[1]{\mathrm{reg}\left(#1\right)}
%\newcommand{\sgncup}{\ \hat{\cup} \ }
\newcommand{\Span}{\mathrm{span}}
\newcommand{\timeline}[2]{\phi_{t_0}^{#1}\left( #2 \right)}
\newcommand{\timeBP}[1]{\overleftarrow{#1}}
\newcommand{\timeBPA}[1]{\mathring{\overleftarrow{#1}}}
\newcommand{\streak}[2]{\Psi_{t_0}^{#1}\left(#2\right)}
\newcommand{\timelineA}[2]{\mathring{\phi}_{t_0,#2}^{#1}}
\newcommand{\DRLN}[1]{{\cal D}_{\curve{#1}}}
\newcommand{\DRLLN}[1]{{\cal D}_{\overline{#1}}}
\newcommand{\DRLNA}[1]{\mathring{\cal D}_{\curve{#1}}}
%\newcommand{\oplusDR}{\,\overline{\oplus}\,}
\newcommand{\oplusDR}{\,\bar{\oplus}\,}
\newcommand{\qo}{\hat{q}}
\newcommand{\xo}{\hat{x}}
\newcommand{\yo}{\hat{y}}
\newcommand{\closure}[1]{\textrm{cl}\left(#1\right)}
\newcommand{\vertexSequence}[4]{
  \left( #1 \rightarrow #2 \rightarrow #3 \rightarrow #4 \rightarrow #1\right)}

\newcommand{\ppSpace}{\Pi_{<\kappa,\bm{\xi},\bm{\nu}}}
\newcommand{\pnSpace}{\mathbb{P}_{<\kappa}}
\newcommand{\pnSpaceK}[1]{\mathbb{P}_{#1}}

\newcommand{\Pyr}[2]{\textrm{Pyr}_{\cal{#1}}\left(\mathbf{#2}\right)}

%\pagestyle{plain}
\pagestyle{fancy}
\fancyhf{}
\fancyhead[LE,RO]{\textbf{\thepage}}

\makeatletter
\newcommand\sixteen{\@setfontsize\sixteen{17pt}{6}}
\renewcommand{\maketitle}{\bgroup\setlength{\parindent}{0pt}
\begin{flushleft}
\sixteen\bfseries \@title
\medskip
\end{flushleft}
\textit{\@author}
\egroup}
\makeatother

\CTEXsetup[format={\Large\bfseries}]{section}

\title{Modern C++ Design}


\begin{document}
\maketitle

\section{Policy-Based Class Design}

In brief, policy-based class design
fosters assembling a class with complex behavior out of many little classes 
(called policies), each of which
takes care of only one behavioral or structural aspect.

The generic SingletonHolder class \textbf{template} (Chapter
6) uses policies for managing lifetime and thread safety. {SmartPtr} (Chapter 7) 
is built almost entirely from policies. The \textbf{double-dispatch engine} in Chapter 11 
uses policies for selecting various trade-offs. The
generic \textbf{Abstract Factory} implementation in Chapter 9 uses a policy for choosing a
creation method.

\subsection{Failure of the Do-it-all Interface}

Implementing everything under the umbrella of a do-it-all interface is not a good solution, for several
reasons:

\begin{itemize}
    \item Intellectual overhead, sheer size, and inefficiency.
    \item Loss of static type safety. A design should enforce most constraints at compile time. (No two singleton objects)
\end{itemize}

\subsection{Multiple Inheritance to the Rescue?}
For example, the user would build a multi-threaded, reference-counted smart pointer
class by inheriting some 
\texttt{BaseSmartPtr} class and two classes: \texttt{MultiThreaded}
and  \texttt{RefCounted}. Any experienced class 
designer knows that such a naive design does not work.

The problems with assembling separate features by using
multiple inheritance are as follows:
\begin{itemize}
\item \textbf{Mechanics}. There is no boilerplate code to assemble the
  inherited components in a controlled manner. \textbf{The language applies
  simple superposition in combining the base classes and establishes a
  set of simple rules for accessing their members.  }
\item \textbf{Type information}. The base classes do not have enough type
  information to carry out their tasks. 
\item \textbf{State manipulation}. Various behavioral aspects implemented with
  base classes must manipulate the same state. This means that they
  must use virtual inheritance to inherit a base class that holds the
  state. This complicates the design and makes it more rigid because
  the premise was that user classes inherit library classes, not vice versa.
\end{itemize}

\subsection{Templates}

Benefits:
\begin{itemize}
\item Class templates are customizable in ways not supported by
  regular classes.
\item for class templates with multiple parameters, you can use
  partial template specialization.
\end{itemize}

As soon as you try to implement such designs, you stumble upon several
problems that are not self-evident:
\begin{itemize}
\item You cannot specialize structure.
\item Specialization of member functions does not scale: you cannot
  specialize individual member functions for templates with multiple
  template parameters.
\item The library writer cannot provide multiple \textbf{default} values.
\end{itemize}

Multiple inheritance and templates foster complementary trade-offs:
\begin{itemize}
\item Multiple inheritance has scarce mechanics; templates have rich
mechanics
\item Multiple inheritance loses type information, which abounds
in templates.
\item Specialization of templates does not scale, but multiple
inheritance scales quite nicely.
\item You can provide only one default for
a template member function, but you can write an unbounded 
number of base classes.
\end{itemize}

\subsection{Policy Classes}

A \textbf{policy} defines a class interface or a class template interface. The
interface consists of one or all of the following: \textbf{inner type
definitions, member functions, and member variables}. The
implementations of a policy are called \textbf{policy classes}. Policy classes
are not intended for stand-alone use; instead, they are inherited by,
or contained within, other classes.

\begin{verbatim}
template <class T>
struct OpNewCreator{
  static T* Create(){
    return new T;
  }
};

template <class T>
struct MallocCreator{
  static T* Create(){
    void* buf = std::malloc(sizeof(T));
    if (!buf) return 0;
    return new(buf) T;
  }
};

template <class T>
struct PrototypeCreator{
  PrototypeCreator(T* pObj = 0):pPrototype_(pObj){}
  T* Create(){
    return pPrototype_ ? pPrototype_->Clone() : 0;
  }
  T* GetPrototype() { return pPrototype_; }
  void SetPrototype(T* pObj) { pPrototype_ = pObj; }
private:
  T* pPrototype_;
};
\end{verbatim}

The classes that use one or more policies are called hosts or
\textbf{host classes}.

\begin{verbatim}
template <class CreationPolicy>
class WidgetManager : public CreationPolicy{
  ...
};
typedef WidgetManager< OpNewCreator<Widget> > MyWidgetMgr;
\end{verbatim}

\textbf{It is the user of \texttt{WidgetManager} who chooses the creation
  policy}. This is the gist of policy-based class design.

\subsection{Implementing Policy Classes with Template Template
  Parameters}

The policy's template argument is redundant. In this case, we can use
\textbf{template template parameters} for specifying policies, as shown in the
following:
\begin{verbatim}
template <template <class> class CreationPolicy = OpNewcreator>
class WidgetManager : public CreationPolicy<Widget>{
  ...
};
typedef WidgetManager<OpNewCreator> MyWidgetMgr;
\end{verbatim}

Using template template parameters with policy classes is not simply a
matter of convenience; sometimes, it is essential that the host class
have access to the template so that the host can instantiate it with a
different type. For example:

\begin{verbatim}
template <template <class> class CreationPolicy = OpNewcreator>
class WidgetManager : public CreationPolicy<Widget>{
  void DoSomething(){
    Gadget* pW = CreationPolicy<Gadget>().Create();
  }
};
\end{verbatim}

Benefits of using policies:
\begin{itemize}
\item you can change policies from theoutside as easily as changing a
  template argument when you instantiate \texttt{WidgetManager}.
\item you can provide your own policies that are specific to your
  concrete application.
\item Policies allow you to generate designs by combining simple
  choices in a typesafe manner.
\item the binding between a host class and its policies is done at
  compile time, the code is tight and efficient, comparable to its
  handcrafted equivalent.
\end{itemize}

\subsection{Destructors of Policy Classes}

The user can automatically convert a host class to a policy and later
\texttt{delete} that pointer. Unless the policy class defines a virtual
destructor, applying delete to a pointer to the policy class has
undefined behavior.

Defining a virtual destructor for a policy, however, works against its
static nature and hurts performance. The lightweight, effective
solution that policies should use is to define a nonvirtual protected
destructor:
\begin{verbatim}
template <class T>
struct OpNewCreator{
protected:
  ~OpNewCreator() {}
};
\end{verbatim}

Because the destructor is protected, \textbf{only derived classes can destroy
policy objects}, so it's impossible for outsiders to apply delete to a
pointer to a policy class.

\subsection{Enriched Policies}

The \texttt{Creator} policy prescribes only one member function,
\texttt{Create}. However, \texttt{PrototypeCreator} defines two more functions:
\texttt{GetPrototype} and \texttt{SetPrototype}.

A user who uses a prototype-based Creator policy class can write the
following code: 
\begin{verbatim}
typedef WidgetManager<PrototypeCreator> MyWidgetManager;

Widget* pPrototype = ...;
MyWidgetManager mgr;
mgr.SetPrototype(pPrototype);
\end{verbatim}

If the user later decides to use a creation policy that does not
support prototypes, \textbf{the compiler pinpoints the spots where the
prototype-specific interface was used.} This is exactly what should be
expected from a sound design.

\subsection{Optional Functionality Through Incomplete Instantiation}

If a member function of a class template is never used, \textbf{it is not even
instantiated—the compiler does not look at it at all}, except perhaps
for syntax checking.

\begin{verbatim}
template <template <class> class CreationPolicy>
class WidgetManager : public CreationPolicy<Widget>{
  void SwitchPrototype(Widget* pNewPrototype){
    CreationPolicy<Widget>& myPolicy = *this;
    delete myPolicy.GetPrototype();
    myPolicy.SetPrototype(pNewPrototype);
  }
};
\end{verbatim}
The resulting context is very interesting:
\begin{itemize}
\item If the user instantiates \texttt{WidgetManager} with a Creator
  policy class that does not support prototypes and tries to use
  \texttt{SwitchPrototype}, a compile-time error occurs.
\item  If the user instantiates \texttt{WidgetManager} with a Creator
  policy class that does not support prototypes and does not try to
  use \texttt{SwitchPrototype}, the program is valid.
\end{itemize}
This all means that \texttt{WidgetManager} can benefit from optional
enriched interfaces but still work correctly with poorer interfaces.

\subsection{Compatible and Incompatible Policies}

Suppose you create two instantiations of \texttt{SmartPtr }:
\texttt{FastWidgetPtr}, a pointer with out checking, and 
\texttt{SafeWidgetPtr}, a pointer with checking before dereference.
It is natural to accept the conversion from \texttt{FastWidgetPtr} to
\texttt{SafeWidgetPtr}, but freely converting \texttt{SafeWidgetPtr}
objects to \texttt{FastWidgetPtr} objects is dangerous. 

The best, most scalable way to implement conversions between policies
is to initialize and copy \texttt{SmartPtr} objects policy by policy,
as shown below:

\begin{verbatim}
template<class T,template <class> class CheckingPolicy>
class SmartPtr : public CheckingPolicy<T>{
  template<class T1,template <class> class CP1,>
  SmartPtr(const SmartPtr<T1, CP1>& other)
    : pointee_(other.pointee_), CheckingPolicy<T>(other){ ... }
};
\end{verbatim}

When you initialize a \texttt{SmartPtr<Widget, EnforceNotNull> } with a
\texttt{SmartPtr<ExtendedWidget, NoChecking>}. The compiler tries to
match \texttt{SmartPtr<ExtendedWidget, NoChecking>} to
\texttt{EnforceNotNull}'s constructors.

If \texttt{EnforceNotNull}
implements a \textbf{constructor} that accepts a \texttt{NoChecking} object,
then the compiler matches that constructor. If \texttt{NoChecking}
implements a \textbf{conversion} operator to \texttt{EnforceNotNull}, that 
conversion is invoked. In any other case, the code fails to compile.

Although conversions from \texttt{NoChecking} to
\texttt{EnforceNotNull} and even vice versa are quite sensible, some
conversions don't make any sense at all.  As soon as you try to
confine a pointer to another ownership policy, you break the invariant
that makes reference counting work.

In conclusion, conversions that change the ownership policy should not
be allowed implicitly and should be treated with maximum care.

\subsection{ Decomposing a Class into Policies}

Two policies that do not interact with each other are orthogonal. By
this definition, the Array and the Destroy policies are not
orthogonal.

Nonorthogonal policies are an imperfection you should strive to
avoid. \textbf{They reduce compile-time type safety and complicate the design
  of both the host class and the policy classes.}

If you must use nonorthogonal policies, you can minimize dependencies
by passing a policy class as an argument to another policy class's
template function. However, this decreases encapsulation.
%%% Local Variables:
%%% mode: latex
%%% TeX-master: "../DesignPattern.tex"
%%% End:


\newpage
\section{Techniques}

\subsection{Compile-Time Assertions}

\textbf{C++17 provides \texttt{static\_assert}}.

The simplest solution to compile-time assertions works in C as well as in
C++, relies on the fact that a zero-length array is illegal.

\begin{verbatim}
#define STATIC_CHECK(expr) { char unnamed[(expr) ? 1 : 0]; }
template <class To, class From>
To safe_reinterpret_cast(From from){
  STATIC_CHECK(sizeof(From) <= sizeof(To));
  return reinterpret_cast<To>(from);
}
void* somePointer = ...;
char c = safe_reinterpret_cast<char>(somePointer);
\end{verbatim}

The problem with this approach is that the error message you receive
is not terribly informative. Error messages have no rules that they
must obey; it's all up to the compiler.

A better solution is to rely on a template with an informative name;
with luck, the compiler will mention the name of that template in the
error message.

\begin{verbatim}
template<bool> struct CompileTimeError;
template<> struct CompileTimeError<true> {};
#define STATIC_CHECK(expr) \
(CompileTimeError<(expr) != 0>())
\end{verbatim}

If you try to instantiate
\texttt{CompileTimeError<false>}, the compiler utters a message such
as "Undefined specialization \texttt{CompileTimeError<false>}." This
message is a slightly better hint that the error is intentional and
not a compiler or a program bug.

Actually, the name \texttt{CompileTimeError} is no longer suggestive
in the new context. \textbf{The ellipsis means the constructor accepts anything.}

\begin{verbatim}
template<bool> struct CompileTimeChecker{
  CompileTimeChecker(...);
};
template<> struct CompileTimeChecker<false> { };
#define STATIC_CHECK(expr, msg) {\
  class ERROR_##msg {}; \
  (void)sizeof(CompileTimeChecker<(expr) != 0>((ERROR_##msg())));\
}

template <class To, class From>
To safe_reinterpret_cast(From from){
  STATIC_CHECK(sizeof(From) <= sizeof(To),Destination_Type_Too_Narrow);
  return reinterpret_cast<To>(from);
}
void* somePointer = ...;
char c = safe_reinterpret_cast<char>(somePointer);
\end{verbatim}

After macro preprocessing, the code of \texttt{safe\_reinterpret\_cast}
expands to the following:

\begin{verbatim}
template <class To, class From>
To safe_reinterpret_cast(From from){
  class ERROR_Destination_Type_Too_Narrow {};
  (void)sizeof(
    CompileTimeChecker<(sizeof(From) <= sizeof(To))>(
      ERROR_Destination_Type_Too_Narrow()));
  return reinterpret_cast<To>(from);
}
\end{verbatim}

The \texttt{CompileTimeChecker<true>} specialization has a constructor
that accept anything; it's an ellipsis function. If the comparison
between sizes evaluates to false, a decent compiler outputs an error
message such as "Error: Cannot convert
\texttt{ERROR\_Destination\_Type\_Too\_Narrow} to
\texttt{CompileTimeChecker <false>}.

\subsection{Partial Template Specialization}

\begin{verbatim}
template <class Window, class Controller>
class Widget{
  ... generic implementation ...
};

// Partial specialization of Widget
template <class Window>
class Widget<Window, MyController>{
  ... partially specialized implementation ...
};

template <class ButtonArg>
class Widget<Button<ButtonArg>, MyController>{
  ... further specialized implementation ...
};
\end{verbatim}

Unfortunately, partial template specialization does not apply to
functions—be they member or nonmember—which somewhat reduces the
flexibility and the granularity of what you can do:
\begin{itemize}
\item Although you can \textbf{totally specialize} member functions of
  a class template, you cannot \textbf{partially specialize} member
  functions.
\item You cannot partially specialize namespace-level (nonmember)
  template functions. The closest thing to partial specialization for
  namespace-level template functions is overloading (not for changing the
  return value or for internally used type).
\end{itemize}

\begin{verbatim}
template <class T, class U> T Fun(U obj); // primary template
template <class U> void Fun<void, U>(U obj); // illegal partial specialization
template <class T> T Fun (Window obj); // legal (overloading)
\end{verbatim}

\subsection{Local Classes}

\textbf{Local classes cannot define static member variables and cannot
  access nonstatic local variables}. What makes local classes truly
interesting is that you can use them in template functions. \textbf{Local
classes defined inside template functions can use the template
parameters of the enclosing function.}

\begin{verbatim}
class Interface{
public:
virtual void Fun() = 0;
};
template <class T, class P>
Interface* MakeAdapter(const T& obj, const P& arg){
  class Local : public Interface{
  public:
    Local(const T& obj, const P& arg): obj_(obj), arg_(arg) {}
    virtual void Fun(){
      obj_.Call(arg_);
    }
  private:
     T obj_;
     P arg_;
  };
  return new Local(obj, arg);
}
\end{verbatim}

It can be easily proven that any idiom that uses a local class can be
implemented using a template class outside the function. On the other
hand, local classes can simplify implementations and improve locality
of symbols.

Local classes do have a unique feature, though: They are
\textbf{final}. Outside users cannot derive from a class hidden in a
function. Without local classes, you'd have to add an unnamed
namespace in a separate translation unit.

\subsection{Mapping Integral Constants to Types}

\begin{verbatim}
template <int v>
struct Int2Type{
enum { value = v };
};
\end{verbatim}

\texttt{Int2Type} generates a distinct type for each distinct constant
integral value passed.  You can use \texttt{Int2Type} whenever you
need to "typify" an integral constant quickly. This way you can 
\textbf{select different functions, depending on the result of a
  compile-time calculation.} Effectively, you \textbf{achieve static
  dispatching on a constant integral value. }

For dispatching at runtime, you can use simple \texttt{if-else}
statements or the \texttt{switch} statement.  However, the
\texttt{if-else} statement requires both branches to compile
successfully, even when the condition tested by \texttt{if} is known
at compile time.

\begin{verbatim}
template <typename T, bool isPolymorphic>
class NiftyContainer{
  void DoSomething(){
    T* pSomeObj = ...;
    if (isPolymorphic){
      T* pNewObj = pSomeObj->Clone();
      ... polymorphic algorithm ...
    }
    else{
      T* pNewObj = new T(*pSomeObj);
      ... nonpolymorphic algorithm ...
    }
  }
};
\end{verbatim}

The polymorphic algorithm uses \texttt{pObj->Clone()},
\texttt{NiftyContainer::DoSomething }does not compile for any type
that doesn't define a member function \texttt{Clone()}.

If \texttt{T} has disabled its copy constructor (by making it
private), if \texttt{T} is a polymorphic type and the nonpolymorphic
code branch attempts \texttt{new T(*pObj)}, the code might fail to
compile.

\begin{verbatim}
template <typename T, bool isPolymorphic>
class NiftyContainer{
private:
  void DoSomething(T* pObj, Int2Type<true>){
    T* pNewObj = pObj->Clone();
    ... polymorphic algorithm ...
  }
  void DoSomething(T* pObj, Int2Type<false>){
    T* pNewObj = new T(*pObj);
    ... nonpolymorphic algorithm ...
  }
public:
  void DoSomething(T* pObj){
    DoSomething(pObj, Int2Type<isPolymorphic>());
  }
};
\end{verbatim}

\textbf{There is another solution, \texttt{if constexpr()}, the new
  feature provided by c++17.}

\subsection{Type-to-Type Mapping}

\begin{verbatim}
template <class T, class U>
T* Create(const U& arg){
  return new T(arg);
}
\end{verbatim}

If objects of type \texttt{Widget} are untouchable legacy code and must
take two arguments upon construction, the second being a fixed value
such as -1. How can you specialize Create so that it treats \texttt{Widget}
differently from all other types with a uniform interface?

\begin{verbatim}
// Illegal code — don't try this at home
template <class U>
Widget* Create<Widget, U>(const U& arg){
  return new Widget(arg, -1);
}

// rely on overloading
template <class T, class U>
T* Create(const U& arg, T /* dummy */){
  return new T(arg);
}
template <class U>
Widget* Create(const U& arg, Widget /* dummy */){
  return new Widget(arg, -1);
}
\end{verbatim}

Such a solution would incur the overhead of constructing an
arbitrarily complex object that remains unused.

\begin{verbatim}
template <typename T>
struct Type2Type{
  typedef T OriginalType;
};
template <class T, class U>
T* Create(const U& arg, Type2Type<T>){
  return new T(arg);
}
template <class U>
Widget* Create(const U& arg, Type2Type<Widget>){
  return new Widget(arg, -1);
}
// Use Create()
String* pStr = Create("Hello", Type2Type<String>());
Widget* pW = Create(100, Type2Type<Widget>());
\end{verbatim}

\subsection{Type Selection}

Sometimes generic code needs to select one type or another, depending
on a Boolean constant.

You might want to use an \texttt{std::vector} as your back-end
storage. Obviously, you cannot store polymorphic types by value, so
you must store pointers. On the other hand, you might want to store
nonpolymorphic types by value, because this is more efficient.

\begin{verbatim}
template <typename T, bool isPolymorphic>
struct NiftyContainerValueTraits{
  typedef T* ValueType;
};
template <typename T>
struct NiftyContainerValueTraits<T, false>{
  typedef T ValueType;
};
template <typename T, bool isPolymorphic>
class NiftyContainer{
  typedef NiftyContainerValueTraits<T, isPolymorphic> Traits;
  typedef typename Traits::ValueType ValueType;
};
\end{verbatim}

This way of doing things is unnecessarily clumsy. Moreover, it doesn't
scale: For each type selection, you must define a new traits class
template.

\begin{verbatim}
template <bool flag, typename T, typename U>
struct Select{
  typedef T Result;
};
template <typename T, typename U>
struct Select<false, T, U>{
  typedef U Result;
};

template <typename T, bool isPolymorphic>
class NiftyContainer{
  typedef typename Select<isPolymorphic, T*, T>::Result ValueType;
}
\end{verbatim}

\subsection{Detecting Convertibility and Inheritance at Compile Time}

 In a generic function, you can rely on an optimized algorithm if a
 class implements a certain interface. Discovering this at compile
 time means not having to use \texttt{dynamic\_cast}, which is costly
 at runtime.

 Detecting inheritance relies on a more general mechanism, that of
 detecting convertibility. The more general problem is, How can you
 detect whether an arbitrary type \texttt{T} supports automatic
 conversion to an arbitrary type \texttt{U}?

 There is a surprising amount of power in \texttt{sizeof}: You can
 apply \texttt{sizeof} to any expression, no matter how complex, and
 \texttt{sizeof} \textbf{returns its size without actually evaluating
   that expression at runtime. }

 The idea of conversion detection relies on using \texttt{sizeof} in
 conjunction with overloaded functions. We provide two overloads of a
 function: \textbf{One accepts the type to convert to (\texttt{U}), and the
 other accepts just about anything else. } If the function that
accepts a \texttt{U} gets called, we know that \texttt{T} is
convertible to \texttt{U}.

\begin{verbatim}
typedef char Small;
class Big { char dummy[2]; };
Small Test(U);
Big Test(...);
const bool convExists = sizeof(Test(T())) == sizeof(Small);
\end{verbatim}
Passing a C++ object to a function with ellipses has undefined
results, but this doesn't matter. Nothing actually calls the
function. It's not even implemented. Recall that \texttt{sizeof} does
not evaluate its argument.

There is one little problem. If \texttt{T} makes its default constructor
private, the expression \texttt{T()} fails to compile. Fortunately,
there is a simple solution, just use a strawman function
returning a \texttt{T}.  \texttt{MakeT} and \texttt{Test} not only
don't do anything but don't even really exist at all.

\begin{verbatim}
template <class T, class U>
class Conversion{
  typedef char Small;
  class Big { char dummy[2]; };
  static Small Test(U);
  static Big Test(...);
  static T MakeT(); // not implemented
public:
  enum { exists = sizeof(Test(MakeT())) == sizeof(Small) };
};
cout << Conversion<size_t, vector<int> >::exists << ' ';
// return 0, because that constructor is explicit.
\end{verbatim}

We can implement one more constant inside
\texttt{Conversion::sameType}, which is true if \texttt{T} and
\texttt{U} represent the same type:

\begin{verbatim}
template <class T, class U>
class Conversion{
  ... as above ...
  enum { sameType = false };
};
template <class T>
class Conversion<T, T>{
public:
  enum { exists = 1, sameType = 1 };
};
#define SUPERSUBCLASS(T, U) \
(Conversion<const U*, const T*>::exists && \
!Conversion<const T*, const void*>::sameType)
\end{verbatim}

There are only three cases in which \texttt{const U*} converts
implicitly to \texttt{const T*}: 
\begin{enumerate}
\item \texttt{T} is the same type as \texttt{U}
\item \texttt{T} is an unambiguous public base of \texttt{U}
\item \texttt{T} is \texttt{void}.
\end{enumerate}

Using \texttt{const} in \texttt{SUPERSUBCLASS}, we're always on the
safe side, we don't want the conversion test to fail due to
\texttt{const} issues.

Why use \texttt{SUPERSUBCLASS} and not the cuter \texttt{BASE\_OF} or
\texttt{INHERITS}? Think with \texttt{INHERITS(T, U)} it was a
constant struggle to say which way the test worked.

\subsection{A Wrapper Around \texttt{type\_info} }

tandard C++ provides the \texttt{std::type\_info} class, which gives you the
ability to investigate object types at runtime. You typically use
\texttt{type\_info} in conjunction with the \texttt{typeid}
operator. The \texttt{typeid} operator returns a reference to a
\texttt{type\_info} object:

\begin{verbatim}
void Fun(Base* pObj){
  // Compare the two type_info objects corresponding to the type of *pObj and Derived
  if (typeid(*pObj) == typeid(Derived)){
    ... aha, pObj actually points to a Derived object ...
  }
}
\end{verbatim}

In addition to supporting the comparison operators \texttt{operator==}
and \texttt{operator!=}, \texttt{type\_info} provides two more
functions:

\begin{enumerate}
\item The \texttt{name} member function returns a textual
  representation of a type, in the form of \texttt{const char*}.
\item he before member function introduces an implementation's
  collation ordering relationship for \texttt{type\_info} objects.
\item The \texttt{type\_info} class disables the copy constructor and
  assignment operator, which makes storing \texttt{type\_info} objects
  impossible.
\item The objects returned by \texttt{typeid} have static storage, so
  you don't have to worry about lifetime issues.  
\end{enumerate}

You do have to worry about pointer identity, the standard does not
guarantee that each invocation returns a reference to the same
\texttt{type\_info} object. Consequently, you cannot compare pointers
to \texttt{type\_info} objects. What you should do is to store pointers
to \texttt{type\_info} objects and compare them by applying
\texttt{type\_info::operator==} to the dereferenced pointers.

If you want to use STL's ordered containers with \texttt{type\_info},
you must write a little functor and deal with pointers. All this is
clumsy enough to mandate a wrapper class around \texttt{type\_info}
that stores a pointer to a \texttt{type\_info} object and provides:
\begin{enumerate}
\item All member functions of \texttt{type\_info}
\item Value semantics (public copy constructor and assignment
  operator)
\item Seamless comparisons by defining \texttt{operator<} and
  \texttt{operator==}
\end{enumerate}

\begin{verbatim}
class TypeInfo{
public:
  // Constructors/destructors
  TypeInfo(); // needed for containers
  TypeInfo(const std::type_info&);
  TypeInfo(const TypeInfo&);
  TypeInfo& operator=(const TypeInfo&);
  // Compatibility functions
  bool before(const TypeInfo&) const;
  const char* name() const;
private:
  const std::type_info* pInfo_;
};
// Comparison operators
bool operator==(const TypeInfo&, const TypeInfo&);
bool operator!=(const TypeInfo&, const TypeInfo&);
bool operator<(const TypeInfo&, const TypeInfo&);
bool operator<=(const TypeInfo&, const TypeInfo&);
bool operator>(const TypeInfo&, const TypeInfo&);
bool operator>=(const TypeInfo&, const TypeInfo&);

void Fun(Base* pObj){
  TypeInfo info = typeid(Derived);
  if (typeid(*pObj) == info){
    ... pBase actually points to a Derived object ...
  }
}
\end{verbatim}

\textbf{The cloning factory in Chapter 8 and one double-dispatch
  engine in Chapter 11 put \texttt{TypeInfo} to good use. }

\subsection{\texttt{NullType} and \texttt{EmptyType}}

\begin{verbatim}
class NullType {};
struct EmptyType {};
\end{verbatim}

You can use \texttt{NullType} for cases in which a type must be there
syntactically but doesn't have a semantic sense. You can use
\texttt{EmptyType} as a default ("don't care") type for a template.

\subsection{Type Traits}

\textbf{Traits are a generic programming technique that allows
  compile-time decisions to be made based on types, much as you would
  make runtime decisions based on values.}

\subsubsection{ Implementing Pointer Traits}

\begin{verbatim}
template <typename T>
class TypeTraits{
private:
  template <class U> 
  struct PointerTraits{
    enum { result = false };
    typedef NullType PointeeType;
  };
  template <class U> 
  struct PointerTraits<U*>{
    enum { result = true };
    typedef U PointeeType;
  };
public:
  enum { isPointer = PointerTraits<T>::result };
  typedef PointerTraits<T>::PointeeType PointeeType;
};

const bool iterIsPtr = TypeTraits<vector<int>::iterator>::isPointer;
cout << "vector<int>::iterator is " << iterIsPtr ? "fast" : "smart" << '\n';
\end{verbatim}

Similarly, TypeTraits implements an \texttt{isReference} constant and
a \texttt{ReferencedType} type definition.

Detection of pointers to members is a bit different. The
specialization needed is as follows:
\begin{verbatim}
template <typename T>
class TypeTraits{
private:
  template <class U> 
  struct PToMTraits{
    enum { result = false };
  };
template <class U, class V>
  struct PToMTraits<U V::*>{
    enum { result = true };
  };
public:
  enum { isMemberPointer = PToMTraits<T>::result };
};
\end{verbatim}

\subsubsection{Detection of Fundamental Types}

\texttt{TypeTraits<T>} implements an \texttt{isStdFundamental}
compile-time constant that says whether or not \texttt{T} is a
standard fundamental type.

In Section 3, we will know an \texttt{TypeList} and the expression
\begin{verbatim}
TL::IndexOf<T, TYPELIST_nn(comma-separated list of types)>::value
\end{verbatim}
returns the zero-based position of \texttt{T} in the list, or –1 if
\texttt{T} does not figure in the list.

\begin{verbatim}
template <typename T>
class TypeTraits
{
... as above ...
public:
  typedef TYPELIST_4(unsigned char, unsigned short int, unsigned int, unsigned long int) UnsignedInts;
  typedef TYPELIST_4(signed char, short int, int, long int) SignedInts;
  typedef TYPELIST_3(bool, char, wchar_t) OtherInts;
  typedef TYPELIST_3(float, double, long double) Floats;
  enum { isStdUnsignedInt = TL::IndexOf<T, UnsignedInts>::value >= 0 };
  enum { isStdSignedInt = TL::IndexOf<T, SignedInts>::value >= 0 };
  enum { isStdIntegral = isStdUnsignedInt || isStdSignedInt || TL::IndexOf <T, OtherInts>::value >= 0 };
  enum { isStdFloat = TL::IndexOf<T, Floats>::value >= 0 };
  enum { isStdArith = isStdIntegral || isStdFloat };
  enum { isStdFundamental = isStdArith || isStdFloat || Conversion<T, void>::sameType };
  ...
};
\end{verbatim}

\subsubsection{Optimized Parameter Types}

Given an arbitrary type \texttt{T}, what is the most efficient way of
passing and accepting objects of type \texttt{T} as arguments to
functions? In general, the most efficient way is to pass elaborate
types by reference and scalar types by value.

A detail that must be carefully handled is that C++ does not allow
references to references. Thus, if \texttt{T} is already a reference,
you should not add one more reference to it.

\begin{verbatim}
template <typename T>
class TypeTraits{
  ... as above ...
public:
  typedef Select<isStdArith || isPointer || isMemberPointer, T,ReferencedType&>::Result ParameterType;
};
\end{verbatim}

\subsubsection{Stripping Qualifiers}

\begin{verbatim}
template <typename T>
class TypeTraits{
  ... as above ...
private:
  template <class U> struct UnConst{
    typedef U Result;
  };
  template <class U> struct UnConst<const U>{
    typedef U Result;
  };
public:
  typedef UnConst<T>::Result NonConstType;
};
\end{verbatim}

\subsubsection{Using \texttt{TypeTraits}}

\begin{verbatim}
enum CopyAlgoSelector { Conservative, Fast };
// Conservative routine-works for any type
template <typename InIt, typename OutIt>
OutIt CopyImpl(InIt first, InIt last, OutIt result, Int2Type<Conservative>){
  for (; first != last; ++first, ++result)
  *result = *first;
  return result;
}
// Fast routine-works only for pointers to raw data
template <typename InIt, typename OutIt>
OutIt CopyImpl(InIt first, InIt last, OutIt result, Int2Type<Fast>){
  const size_t n = last-first;
  BitBlast(first, result, n * sizeof(*first));
  return result + n;
}
template <typename InIt, typename OutIt>
OutIt Copy(InIt first, InIt last, OutIt result){
  typedef TypeTraits<InIt>::PointeeType SrcPointee;
  typedef TypeTraits<OutIt>::PointeeType DestPointee;
  enum { copyAlgo = 
         TypeTraits<InIt>::isPointer &&
         TypeTraits<OutIt>::isPointer &&
         TypeTraits<SrcPointee>::isStdFundamental &&
         TypeTraits<DestPointee>::isStdFundamental &&
         sizeof(SrcPointee) == sizeof(DestPointee) ? Fast : Conservative };
  return CopyImpl(first, last, result, Int2Type<copyAlgo>);
}
\end{verbatim}

The drawback of Copy is that it doesn't accelerate everything that
could be accelerated. For example, you  might have a plain C-like
struct containing nothing but primitive data—a so-called plain old
data, or POD, structure.
\begin{verbatim}
template <typename T> 
struct SupportsBitwiseCopy{
  enum { result = TypeTraits<T>::isStdFundamental };
};
template<> 
struct SupportsBitwiseCopy<MyType>{
  enum { result = true };
};
template <typename InIt, typename OutIt>
OutIt Copy(InIt first, InIt last, OutIt result, Int2Type<true>){
  typedef TypeTraits<InIt>::PointeeType SrcPointee;
  typedef TypeTraits<OutIt>::PointeeType DestPointee;
  enum { useBitBlast =
         TypeTraits<InIt>::isPointer &&
         TypeTraits<OutIt>::isPointer &&
         SupportsBitwiseCopy<SrcPointee>::result &&
         SupportsBitwiseCopy<DestPointee>::result &&
         sizeof(SrcPointee) == sizeof(DestPointee) };
  return CopyImpl(first, last, Int2Type<useBitBlast>);
}
\end{verbatim}

\subsubsection{Summary}

The most important point is that the compiler always find the best
match of template specialization.
%%% Local Variables:
%%% mode: latex
%%% TeX-master: "../DesignPattern"
%%% End:



\section{Typelists}

\subsection{The need for Typelists}

If you want to generalize the concept of Abstract Factory and put it
into a library,  you have to make it possible for the user to create
factories of arbitrary collections of types.

\begin{itemize}
\item . In the Abstract Factory case, although the abstract base class
  is quite simple, you can get a nasty amount of code duplication when
  implementing various concrete factories.
\item You cannot easily manipulate the member functions of
  \texttt{WidgetFactory} \textbf{because virtual functions cannot be
    templates.}
\item We wish it would be nice if we could create a
\texttt{WidgetFactory} by passing a parameter list to an
\texttt{AbstractFactory} template and we coul have  a template-like
invocation for various \texttt{CreateXxx} functions, such as
\texttt{Create<Window>()}.
\end{itemize}

The definition and algorithm of \texttt{Typelist} is the same as
\texttt{std::Tuple}

\begin{verbatim}
template <class T, class U>
struct Typelist{
  typedef T Head;
  typedef U Tail;
};
typedef Typelist<int, NullType> OneTypeOnly;
#define TYPELIST_1(T1) Typelist<T1, NullType>
#define TYPELIST_2(T1, T2) Typelist<T1, TYPELIST_1(T2) >
#define TYPELIST_3(T1, T2, T3) Typelist<T1, TYPELIST_2(T2, T3) >
...
\end{verbatim}

There is a lot of utility algorithms of Typelist:
\begin{itemize}
\item Calculating length
\begin{verbatim}
template <class TList> struct Length;
template <> struct Length<NullType>{
  enum { value = 0 };
};
template <class T, class U>
struct Length< Typelist<T, U> >{
  enum { value = 1 + Length<U>::value };
};
\end{verbatim}
\item  Indexed Access
\begin{verbatim}
template <class Head, class Tail>
struct TypeAt<Typelist<Head, Tail>, 0>{
  typedef Head Result;
};
template <class Head, class Tail, unsigned int i>
struct TypeAt<Typelist<Head, Tail>, i>{
  typedef typename TypeAt<Tail, i - 1>::Result Result;
};
\end{verbatim}
\item Searching Typelists
\begin{verbatim}
template <class T>
struct IndexOf<NullType, T>{
  enum { value = -1 };
};
template <class T, class Tail>
struct IndexOf<Typelist<T, Tail>, T>{
  enum { value = 0 };
};
template <class Head, class Tail, class T>
struct IndexOf<Typelist<Head, Tail>, T>{
private:
  enum { temp = IndexOf<Tail, T>::value };
public:
  enum { value = temp == -1 ? -1 : 1 + temp };
};
\end{verbatim}
\item Appending to Typelist
\begin{verbatim}
template <> struct Append<NullType, NullType>{
  typedef NullType Result;
};
template <class T> struct Append<NullType, T>{
  typedef TYPELIST_1(T) Result;
};
template <class Head, class Tail>
struct Append<NullType, Typelist<Head, Tail> >{
  typedef Typelist<Head, Tail> Result;
};
template <class Head, class Tail, class T>
struct Append<Typelist<Head, Tail>, T>{
  typedef Typelist<Head,typename Append<Tail, T>::Result> Result;
};
\end{verbatim}
\item Erasing a type from Typelist
\begin{verbatim}
template <class T>
struct Erase<NullType, T>{
  typedef NullType Result;
};
template <class T, class Tail>
struct Erase<Typelist<T, Tail>, T>{
  typedef Tail Result;
};
template <class Head, class Tail, class T>
struct Erase<Typelist<Head, Tail>, T>{
  typedef Typelist<Head,typename Erase<Tail, T>::Result> Result;
};
\end{verbatim}
\item Erasing Duplicates
\begin{verbatim}
template <> struct NoDuplicates<NullType>{
typedef NullType Result;
};
template <class Head, class Tail>
struct NoDuplicates< Typelist<Head, Tail> >{
private:
typedef typename NoDuplicates<Tail>::Result L1;
typedef typename Erase<L1, Head>::Result L2;
public:
typedef Typelist<Head, L2> Result;
};
\end{verbatim}
\item Replacing a type in a Typelist
\begin{verbatim}
template <class T, class U>
struct Replace<NullType, T, U>{
  typedef NullType Result;
};
template <class T, class Tail, class U>
struct Replace<Typelist<T, Tail>, T, U>{
  typedef Typelist<U, Tail> Result;
};
template <class Head, class Tail, class T, class U>
struct Replace<Typelist<Head, Tail>, T, U>{
  typedef Typelist<Head,typename Replace<Tail, T, U>::Result> Result;
};
\end{verbatim}
\item Partially Ordering Typelist
\begin{verbatim}
template <class T>
struct MostDerived<NullType, T>{
  typedef T Result;
};
template <class Head, class Tail, class T>
struct MostDerived<Typelist<Head, Tail>, T>{
private:
  typedef typename MostDerived<Tail, T>::Result Candidate;
public:
  typedef typename Select<SUPERSUBCLASS(Candidate, Head), Head, Candidate>::Result Result;
};
template <>
struct DerivedToFront<NullType>{
  typedef NullType Result;
};
template <class Head, class Tail>
struct DerivedToFront< Typelist<Head, Tail> >{
private:
  typedef typename MostDerived<Tail, Head>::Result TheMostDerived;
  typedef typename Replace<Tail, TheMostDerived, Head>::Result L;
public:
  typedef Typelist<TheMostDerived, L> Result;
};
\end{verbatim}
\end{itemize}

\subsection{Class Generation with Typelists}

\begin{verbatim}
template <class TList, template <class> class Unit>
class GenScatterHierarchy;

template <class T1, class T2, template <class> class Unit>
class GenScatterHierarchy<Typelist<T1, T2>, Unit>
: public GenScatterHierarchy<T1, Unit>
, public GenScatterHierarchy<T2, Unit>{
public:
  typedef Typelist<T1, T2> TList;
  typedef GenScatterHierarchy<T1, Unit> LeftBase;
  typedef GenScatterHierarchy<T2, Unit> RightBase;
};

template <class AtomicType, template <class> class Unit>
class GenScatterHierarchy : public Unit<AtomicType>{
  typedef Unit<AtomicType> LeftBase;
};

template <template <class> class Unit>
class GenScatterHierarchy<NullType, Unit>{};

template <class T>
struct Holder{
  T value_;
};
typedef GenScatterHierarchy<TYPELIST_3(int, string, Widget), Holder> WidgetInfo;
WidgetInfo obj;
string name = (static_cast<Holder<string>&>(obj)).value_;
\end{verbatim}

This cast is quite ugly.
\begin{verbatim}
template <class T, class H>
typename Private::FieldTraits<H>::Rebind<T>::Result& Field(H& obj){
  return obj;
}
\end{verbatim}

 If you call \texttt{Field<Widget>(obj) }, the compiler figures out
 that \texttt{Holder<Widget>} is a base class of \texttt{WidgetInfo} and 
 simply returns a reference to that part of the compound object.

 \subsection{Generating Tuples}

\begin{verbatim}
template <class T>
struct TupleUnit{
  T value_;
  operator T&() { return value_; }
  operator const T&() const { return value_; }
};
template <class TList>
struct Tuple : public GenScatterHierarchy<TList, TupleUnit>{};
\end{verbatim}
%%% Local Variables:
%%% mode: latex
%%% TeX-master: "../DesignPattern"
%%% End:


\end{document}
%%% Local Variables:
%%% mode: xelatex
%%% TeX-master: t
%%% End:
